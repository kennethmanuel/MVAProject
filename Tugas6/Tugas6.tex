% Options for packages loaded elsewhere
\PassOptionsToPackage{unicode}{hyperref}
\PassOptionsToPackage{hyphens}{url}
%
\documentclass[
]{article}
\usepackage{lmodern}
\usepackage{amsmath}
\usepackage{ifxetex,ifluatex}
\ifnum 0\ifxetex 1\fi\ifluatex 1\fi=0 % if pdftex
  \usepackage[T1]{fontenc}
  \usepackage[utf8]{inputenc}
  \usepackage{textcomp} % provide euro and other symbols
  \usepackage{amssymb}
\else % if luatex or xetex
  \usepackage{unicode-math}
  \defaultfontfeatures{Scale=MatchLowercase}
  \defaultfontfeatures[\rmfamily]{Ligatures=TeX,Scale=1}
\fi
% Use upquote if available, for straight quotes in verbatim environments
\IfFileExists{upquote.sty}{\usepackage{upquote}}{}
\IfFileExists{microtype.sty}{% use microtype if available
  \usepackage[]{microtype}
  \UseMicrotypeSet[protrusion]{basicmath} % disable protrusion for tt fonts
}{}
\makeatletter
\@ifundefined{KOMAClassName}{% if non-KOMA class
  \IfFileExists{parskip.sty}{%
    \usepackage{parskip}
  }{% else
    \setlength{\parindent}{0pt}
    \setlength{\parskip}{6pt plus 2pt minus 1pt}}
}{% if KOMA class
  \KOMAoptions{parskip=half}}
\makeatother
\usepackage{xcolor}
\IfFileExists{xurl.sty}{\usepackage{xurl}}{} % add URL line breaks if available
\IfFileExists{bookmark.sty}{\usepackage{bookmark}}{\usepackage{hyperref}}
\hypersetup{
  pdftitle={Salespeople-data},
  pdfauthor={Kenneth Manuel},
  hidelinks,
  pdfcreator={LaTeX via pandoc}}
\urlstyle{same} % disable monospaced font for URLs
\usepackage[margin=1in]{geometry}
\usepackage{color}
\usepackage{fancyvrb}
\newcommand{\VerbBar}{|}
\newcommand{\VERB}{\Verb[commandchars=\\\{\}]}
\DefineVerbatimEnvironment{Highlighting}{Verbatim}{commandchars=\\\{\}}
% Add ',fontsize=\small' for more characters per line
\usepackage{framed}
\definecolor{shadecolor}{RGB}{248,248,248}
\newenvironment{Shaded}{\begin{snugshade}}{\end{snugshade}}
\newcommand{\AlertTok}[1]{\textcolor[rgb]{0.94,0.16,0.16}{#1}}
\newcommand{\AnnotationTok}[1]{\textcolor[rgb]{0.56,0.35,0.01}{\textbf{\textit{#1}}}}
\newcommand{\AttributeTok}[1]{\textcolor[rgb]{0.77,0.63,0.00}{#1}}
\newcommand{\BaseNTok}[1]{\textcolor[rgb]{0.00,0.00,0.81}{#1}}
\newcommand{\BuiltInTok}[1]{#1}
\newcommand{\CharTok}[1]{\textcolor[rgb]{0.31,0.60,0.02}{#1}}
\newcommand{\CommentTok}[1]{\textcolor[rgb]{0.56,0.35,0.01}{\textit{#1}}}
\newcommand{\CommentVarTok}[1]{\textcolor[rgb]{0.56,0.35,0.01}{\textbf{\textit{#1}}}}
\newcommand{\ConstantTok}[1]{\textcolor[rgb]{0.00,0.00,0.00}{#1}}
\newcommand{\ControlFlowTok}[1]{\textcolor[rgb]{0.13,0.29,0.53}{\textbf{#1}}}
\newcommand{\DataTypeTok}[1]{\textcolor[rgb]{0.13,0.29,0.53}{#1}}
\newcommand{\DecValTok}[1]{\textcolor[rgb]{0.00,0.00,0.81}{#1}}
\newcommand{\DocumentationTok}[1]{\textcolor[rgb]{0.56,0.35,0.01}{\textbf{\textit{#1}}}}
\newcommand{\ErrorTok}[1]{\textcolor[rgb]{0.64,0.00,0.00}{\textbf{#1}}}
\newcommand{\ExtensionTok}[1]{#1}
\newcommand{\FloatTok}[1]{\textcolor[rgb]{0.00,0.00,0.81}{#1}}
\newcommand{\FunctionTok}[1]{\textcolor[rgb]{0.00,0.00,0.00}{#1}}
\newcommand{\ImportTok}[1]{#1}
\newcommand{\InformationTok}[1]{\textcolor[rgb]{0.56,0.35,0.01}{\textbf{\textit{#1}}}}
\newcommand{\KeywordTok}[1]{\textcolor[rgb]{0.13,0.29,0.53}{\textbf{#1}}}
\newcommand{\NormalTok}[1]{#1}
\newcommand{\OperatorTok}[1]{\textcolor[rgb]{0.81,0.36,0.00}{\textbf{#1}}}
\newcommand{\OtherTok}[1]{\textcolor[rgb]{0.56,0.35,0.01}{#1}}
\newcommand{\PreprocessorTok}[1]{\textcolor[rgb]{0.56,0.35,0.01}{\textit{#1}}}
\newcommand{\RegionMarkerTok}[1]{#1}
\newcommand{\SpecialCharTok}[1]{\textcolor[rgb]{0.00,0.00,0.00}{#1}}
\newcommand{\SpecialStringTok}[1]{\textcolor[rgb]{0.31,0.60,0.02}{#1}}
\newcommand{\StringTok}[1]{\textcolor[rgb]{0.31,0.60,0.02}{#1}}
\newcommand{\VariableTok}[1]{\textcolor[rgb]{0.00,0.00,0.00}{#1}}
\newcommand{\VerbatimStringTok}[1]{\textcolor[rgb]{0.31,0.60,0.02}{#1}}
\newcommand{\WarningTok}[1]{\textcolor[rgb]{0.56,0.35,0.01}{\textbf{\textit{#1}}}}
\usepackage{graphicx}
\makeatletter
\def\maxwidth{\ifdim\Gin@nat@width>\linewidth\linewidth\else\Gin@nat@width\fi}
\def\maxheight{\ifdim\Gin@nat@height>\textheight\textheight\else\Gin@nat@height\fi}
\makeatother
% Scale images if necessary, so that they will not overflow the page
% margins by default, and it is still possible to overwrite the defaults
% using explicit options in \includegraphics[width, height, ...]{}
\setkeys{Gin}{width=\maxwidth,height=\maxheight,keepaspectratio}
% Set default figure placement to htbp
\makeatletter
\def\fps@figure{htbp}
\makeatother
\setlength{\emergencystretch}{3em} % prevent overfull lines
\providecommand{\tightlist}{%
  \setlength{\itemsep}{0pt}\setlength{\parskip}{0pt}}
\setcounter{secnumdepth}{-\maxdimen} % remove section numbering
\ifluatex
  \usepackage{selnolig}  % disable illegal ligatures
\fi

\title{Salespeople-data}
\author{Kenneth Manuel}
\date{3/27/2021}

\begin{document}
\maketitle

\begin{Shaded}
\begin{Highlighting}[]
\FunctionTok{library}\NormalTok{(}\StringTok{"tidyverse"}\NormalTok{)}
\end{Highlighting}
\end{Shaded}

\begin{verbatim}
## -- Attaching packages --------------------------------------- tidyverse 1.3.0 --
\end{verbatim}

\begin{verbatim}
## v ggplot2 3.3.3     v purrr   0.3.4
## v tibble  3.0.6     v dplyr   1.0.4
## v tidyr   1.1.2     v stringr 1.4.0
## v readr   1.4.0     v forcats 0.5.1
\end{verbatim}

\begin{verbatim}
## -- Conflicts ------------------------------------------ tidyverse_conflicts() --
## x dplyr::filter() masks stats::filter()
## x dplyr::lag()    masks stats::lag()
\end{verbatim}

\begin{Shaded}
\begin{Highlighting}[]
\FunctionTok{library}\NormalTok{(}\StringTok{"readxl"}\NormalTok{)}
\FunctionTok{library}\NormalTok{(}\StringTok{"NbClust"}\NormalTok{)}
\FunctionTok{library}\NormalTok{(}\StringTok{"factoextra"}\NormalTok{)}
\end{Highlighting}
\end{Shaded}

\begin{verbatim}
## Warning: package 'factoextra' was built under R version 4.0.4
\end{verbatim}

\begin{verbatim}
## Welcome! Want to learn more? See two factoextra-related books at https://goo.gl/ve3WBa
\end{verbatim}

\begin{Shaded}
\begin{Highlighting}[]
\FunctionTok{library}\NormalTok{(}\StringTok{"MASS"}\NormalTok{)}
\end{Highlighting}
\end{Shaded}

\begin{verbatim}
## 
## Attaching package: 'MASS'
\end{verbatim}

\begin{verbatim}
## The following object is masked from 'package:dplyr':
## 
##     select
\end{verbatim}

\begin{Shaded}
\begin{Highlighting}[]
\FunctionTok{library}\NormalTok{(}\StringTok{"caret"}\NormalTok{)}
\end{Highlighting}
\end{Shaded}

\begin{verbatim}
## Warning: package 'caret' was built under R version 4.0.4
\end{verbatim}

\begin{verbatim}
## Loading required package: lattice
\end{verbatim}

\begin{verbatim}
## 
## Attaching package: 'caret'
\end{verbatim}

\begin{verbatim}
## The following object is masked from 'package:purrr':
## 
##     lift
\end{verbatim}

\begin{Shaded}
\begin{Highlighting}[]
\NormalTok{data }\OtherTok{\textless{}{-}} \FunctionTok{read\_excel}\NormalTok{(}\StringTok{\textquotesingle{}Salespeople{-}data.xlsx\textquotesingle{}}\NormalTok{)}
\FunctionTok{head}\NormalTok{(data)}
\end{Highlighting}
\end{Shaded}

\begin{verbatim}
## # A tibble: 6 x 7
##   Salegrow saleproft Newsale createst Mechtest absttest mathtest
##      <dbl>     <dbl>   <dbl>    <dbl>    <dbl>    <dbl>    <dbl>
## 1     93        96      97.8        9       12        9       20
## 2     88.8      91.8    96.8        7       10       10       15
## 3     95       100.     99          8       12        9       26
## 4    101.      104.    107.        13       14       12       29
## 5    102       108.    103         10       15       12       32
## 6     95.8      97.5    99.3       10       14       11       21
\end{verbatim}

\hypertarget{menentukan-keputusan-berapa-kluster-yang-akan-digunakan}{%
\section{Menentukan keputusan berapa kluster yang akan
digunakan}\label{menentukan-keputusan-berapa-kluster-yang-akan-digunakan}}

\begin{Shaded}
\begin{Highlighting}[]
\FunctionTok{fviz\_nbclust}\NormalTok{(data, kmeans, }\AttributeTok{method =} \StringTok{"wss"}\NormalTok{) }\SpecialCharTok{+}
    \FunctionTok{geom\_vline}\NormalTok{(}\AttributeTok{xintercept =} \DecValTok{2}\NormalTok{ , }\AttributeTok{linetype =} \DecValTok{2}\NormalTok{)}\SpecialCharTok{+}
  \FunctionTok{labs}\NormalTok{(}\AttributeTok{subtitle =} \StringTok{"Elbow method"}\NormalTok{)}
\end{Highlighting}
\end{Shaded}

\includegraphics{Tugas6_files/figure-latex/unnamed-chunk-3-1.pdf}

\begin{Shaded}
\begin{Highlighting}[]
\FunctionTok{fviz\_nbclust}\NormalTok{(data, kmeans, }\AttributeTok{method =} \StringTok{"silhouette"}\NormalTok{)}\SpecialCharTok{+}
  \FunctionTok{labs}\NormalTok{(}\AttributeTok{subtitle =} \StringTok{"Silhouette method"}\NormalTok{)}
\end{Highlighting}
\end{Shaded}

\includegraphics{Tugas6_files/figure-latex/unnamed-chunk-3-2.pdf}

\begin{Shaded}
\begin{Highlighting}[]
\FunctionTok{fviz\_nbclust}\NormalTok{(data, kmeans, }\AttributeTok{nstart =} \DecValTok{25}\NormalTok{,  }\AttributeTok{method =} \StringTok{"gap\_stat"}\NormalTok{, }\AttributeTok{nboot =} \DecValTok{50}\NormalTok{)}\SpecialCharTok{+}
  \FunctionTok{labs}\NormalTok{(}\AttributeTok{subtitle =} \StringTok{"Gap statistic method"}\NormalTok{)}
\end{Highlighting}
\end{Shaded}

\includegraphics{Tugas6_files/figure-latex/unnamed-chunk-3-3.pdf}

Terlihat bahwa metode elbow dan metode silhouette menyarankan untuk
menggunakan 2 cluster sedangkan gap statistics menyarankan untuk
menggunakan 4 cluster.

Karena dua metode yaitu silhouette dan elbow menyarankan untuk
menggunakan 2 cluster maka clustering akan dibagi menjadi \textbf{2
cluster}

\hypertarget{k-means-clustering}{%
\section{K-means clustering}\label{k-means-clustering}}

\begin{Shaded}
\begin{Highlighting}[]
\NormalTok{km.res }\OtherTok{\textless{}{-}} \FunctionTok{eclust}\NormalTok{(data, }\StringTok{"kmeans"}\NormalTok{, }\AttributeTok{k =} \DecValTok{2}\NormalTok{,}
                 \AttributeTok{nstart =} \DecValTok{25}\NormalTok{, }\AttributeTok{graph =} \ConstantTok{FALSE}\NormalTok{)}
\CommentTok{\# k{-}means group number of each observation}
\NormalTok{km.res}\SpecialCharTok{$}\NormalTok{cluster}
\end{Highlighting}
\end{Shaded}

\begin{verbatim}
##  [1] 2 2 2 1 1 2 2 1 1 1 1 1 1 1 1 2 1 1 2 1 2 1 2 1 1 2 1 1 2 1 1 2 2 2 1 1 2 2
## [39] 1 1 1 2 1 2 2 1 2 2 1 1
\end{verbatim}

\begin{Shaded}
\begin{Highlighting}[]
\FunctionTok{fviz\_cluster}\NormalTok{(km.res,  }\AttributeTok{ellipse.type =} \StringTok{"norm"}\NormalTok{)}
\end{Highlighting}
\end{Shaded}

\includegraphics{Tugas6_files/figure-latex/unnamed-chunk-5-1.pdf}

\begin{Shaded}
\begin{Highlighting}[]
\CommentTok{\#fviz\_silhouette(km.res)}
\end{Highlighting}
\end{Shaded}

\hypertarget{hierachical-clustering-dengan-jarak-euclidean-dan-metode-single-linkage}{%
\section{Hierachical clustering dengan jarak euclidean dan metode single
linkage}\label{hierachical-clustering-dengan-jarak-euclidean-dan-metode-single-linkage}}

\begin{Shaded}
\begin{Highlighting}[]
\CommentTok{\# dissimilarity matrix}
\NormalTok{res.dist }\OtherTok{\textless{}{-}} \FunctionTok{dist}\NormalTok{(data, }\AttributeTok{method =} \StringTok{\textquotesingle{}euclidean\textquotesingle{}}\NormalTok{)}
\NormalTok{res.hc }\OtherTok{\textless{}{-}} \FunctionTok{hclust}\NormalTok{(}\AttributeTok{d =}\NormalTok{ res.dist, }\AttributeTok{method =} \StringTok{\textquotesingle{}single\textquotesingle{}}\NormalTok{)}
\NormalTok{hc.cluster }\OtherTok{\textless{}{-}} \FunctionTok{cutree}\NormalTok{(res.hc, }\AttributeTok{k =} \DecValTok{2}\NormalTok{)}
\NormalTok{hc.cluster}
\end{Highlighting}
\end{Shaded}

\begin{verbatim}
##  [1] 1 1 1 1 1 1 1 2 1 1 1 1 1 1 1 1 1 1 1 1 1 1 1 1 1 1 1 1 1 1 1 1 1 1 1 1 1 1
## [39] 1 1 1 1 1 1 1 1 1 1 1 1
\end{verbatim}

\begin{Shaded}
\begin{Highlighting}[]
\FunctionTok{fviz\_dend}\NormalTok{(res.hc, }\AttributeTok{rect =} \ConstantTok{TRUE}\NormalTok{, }\AttributeTok{show\_labels =} \ConstantTok{TRUE}\NormalTok{, }\AttributeTok{cex =} \FloatTok{0.5}\NormalTok{, }\AttributeTok{color\_labels\_by\_k =} \ConstantTok{TRUE}\NormalTok{) }
\end{Highlighting}
\end{Shaded}

\includegraphics{Tugas6_files/figure-latex/unnamed-chunk-7-1.pdf}

\begin{Shaded}
\begin{Highlighting}[]
\CommentTok{\#fviz\_silhouette(res.hc)}
\end{Highlighting}
\end{Shaded}

\hypertarget{comparing-hierachical-clustering-and-k-means-clustering}{%
\section{Comparing hierachical clustering and K-means
clustering}\label{comparing-hierachical-clustering-and-k-means-clustering}}

\begin{Shaded}
\begin{Highlighting}[]
\NormalTok{km.res}\SpecialCharTok{$}\NormalTok{cluster}
\end{Highlighting}
\end{Shaded}

\begin{verbatim}
##  [1] 2 2 2 1 1 2 2 1 1 1 1 1 1 1 1 2 1 1 2 1 2 1 2 1 1 2 1 1 2 1 1 2 2 2 1 1 2 2
## [39] 1 1 1 2 1 2 2 1 2 2 1 1
\end{verbatim}

\begin{Shaded}
\begin{Highlighting}[]
\NormalTok{hc.cluster}
\end{Highlighting}
\end{Shaded}

\begin{verbatim}
##  [1] 1 1 1 1 1 1 1 2 1 1 1 1 1 1 1 1 1 1 1 1 1 1 1 1 1 1 1 1 1 1 1 1 1 1 1 1 1 1
## [39] 1 1 1 1 1 1 1 1 1 1 1 1
\end{verbatim}

Attaching cluster result label to data

\begin{Shaded}
\begin{Highlighting}[]
\NormalTok{cluster.member1 }\OtherTok{\textless{}{-}} \FunctionTok{data.frame}\NormalTok{(}\AttributeTok{cluster =}\NormalTok{ km.res}\SpecialCharTok{$}\NormalTok{cluster)}
\NormalTok{data.c1 }\OtherTok{\textless{}{-}} \FunctionTok{cbind}\NormalTok{(data, cluster.member1)}

\NormalTok{cluster.member2 }\OtherTok{\textless{}{-}} \FunctionTok{data.frame}\NormalTok{(}\AttributeTok{cluster =}\NormalTok{ hc.cluster)}
\NormalTok{data.c2 }\OtherTok{\textless{}{-}} \FunctionTok{cbind}\NormalTok{(data, cluster.member2)}
\end{Highlighting}
\end{Shaded}

\hypertarget{linear-discriminant-analysis-hasil-k-means}{%
\section{Linear discriminant analysis hasil
K-Means}\label{linear-discriminant-analysis-hasil-k-means}}

Dengan mengasumsi bahwa masing - masing variabel berdistribusi normal
univariate.

\hypertarget{data-preparation}{%
\subsection{1. Data preparation}\label{data-preparation}}

\hypertarget{split-data-to-training-and-test-set}{%
\subsubsection{Split data to training and test
set}\label{split-data-to-training-and-test-set}}

\begin{Shaded}
\begin{Highlighting}[]
\CommentTok{\# Split the data into training (80\%) and test set (20\%)}
\FunctionTok{set.seed}\NormalTok{(}\DecValTok{123}\NormalTok{)}
\NormalTok{training.samples }\OtherTok{\textless{}{-}}\NormalTok{ data.c1}\SpecialCharTok{$}\NormalTok{cluster }\SpecialCharTok{\%\textgreater{}\%}
  \FunctionTok{createDataPartition}\NormalTok{(}\AttributeTok{p =} \FloatTok{0.8}\NormalTok{, }\AttributeTok{list =} \ConstantTok{FALSE}\NormalTok{)}
\NormalTok{train.data }\OtherTok{\textless{}{-}}\NormalTok{ data.c1[training.samples, ]}
\NormalTok{test.data }\OtherTok{\textless{}{-}}\NormalTok{ data.c1[}\SpecialCharTok{{-}}\NormalTok{training.samples, ]}
\end{Highlighting}
\end{Shaded}

\hypertarget{normalize-data}{%
\subsubsection{Normalize data}\label{normalize-data}}

\begin{Shaded}
\begin{Highlighting}[]
\CommentTok{\# Estimate preprocessing parameters}
\NormalTok{preproc.param }\OtherTok{\textless{}{-}}\NormalTok{ train.data }\SpecialCharTok{\%\textgreater{}\%} 
  \FunctionTok{preProcess}\NormalTok{(}\AttributeTok{method =} \FunctionTok{c}\NormalTok{(}\StringTok{"center"}\NormalTok{, }\StringTok{"scale"}\NormalTok{))}
\CommentTok{\# Transform the data using the estimated parameters}
\NormalTok{train.transformed }\OtherTok{\textless{}{-}}\NormalTok{ preproc.param }\SpecialCharTok{\%\textgreater{}\%} \FunctionTok{predict}\NormalTok{(train.data)}
\NormalTok{test.transformed }\OtherTok{\textless{}{-}}\NormalTok{ preproc.param }\SpecialCharTok{\%\textgreater{}\%} \FunctionTok{predict}\NormalTok{(test.data)}
\end{Highlighting}
\end{Shaded}

\hypertarget{compute-lda}{%
\subsubsection{Compute LDA}\label{compute-lda}}

\begin{Shaded}
\begin{Highlighting}[]
\NormalTok{model }\OtherTok{\textless{}{-}} \FunctionTok{lda}\NormalTok{(cluster}\SpecialCharTok{\textasciitilde{}}\NormalTok{., }\AttributeTok{data =}\NormalTok{ train.transformed)}
\NormalTok{model}
\end{Highlighting}
\end{Shaded}

\begin{verbatim}
## Call:
## lda(cluster ~ ., data = train.transformed)
## 
## Prior probabilities of groups:
## -0.646418705528501   1.50831031289984 
##                0.7                0.3 
## 
## Group means:
##                      Salegrow  saleproft    Newsale   createst   Mechtest
## -0.646418705528501  0.5430821  0.5055945  0.4661681  0.2414054  0.2591322
## 1.50831031289984   -1.2671915 -1.1797205 -1.0877255 -0.5632794 -0.6046417
##                      absttest   mathtest
## -0.646418705528501  0.3392754  0.5026441
## 1.50831031289984   -0.7916426 -1.1728362
## 
## Coefficients of linear discriminants:
##                   LD1
## Salegrow  -2.08620271
## saleproft -1.33833281
## Newsale   -0.26834511
## createst  -0.06032613
## Mechtest   0.88695151
## absttest  -0.08842310
## mathtest   1.25657273
\end{verbatim}

\hypertarget{make-predictions}{%
\subsection{2. Make predictions}\label{make-predictions}}

\begin{Shaded}
\begin{Highlighting}[]
\NormalTok{predictions }\OtherTok{\textless{}{-}}\NormalTok{ model }\SpecialCharTok{\%\textgreater{}\%} \FunctionTok{predict}\NormalTok{(test.data)}
\FunctionTok{names}\NormalTok{(predictions)}
\end{Highlighting}
\end{Shaded}

\begin{verbatim}
## [1] "class"     "posterior" "x"
\end{verbatim}

\hypertarget{model-accuracy}{%
\subsection{3. Model accuracy}\label{model-accuracy}}

\begin{Shaded}
\begin{Highlighting}[]
\FunctionTok{mean}\NormalTok{(predictions}\SpecialCharTok{$}\NormalTok{class}\SpecialCharTok{==}\NormalTok{test.transformed}\SpecialCharTok{$}\NormalTok{cluster)}
\end{Highlighting}
\end{Shaded}

\begin{verbatim}
## [1] 0.1
\end{verbatim}

\hypertarget{linear-discriminant-analysis-hasil-hierachical-clustering}{%
\section{Linear discriminant analysis hasil Hierachical
clustering}\label{linear-discriminant-analysis-hasil-hierachical-clustering}}

Dengan mengasumsi bahwa masing - masing variabel berdistribusi normal
univariate.

\hypertarget{data-preparation-1}{%
\subsection{1. Data preparation}\label{data-preparation-1}}

\hypertarget{split-data-to-training-and-test-set-1}{%
\subsubsection{Split data to training and test
set}\label{split-data-to-training-and-test-set-1}}

\begin{Shaded}
\begin{Highlighting}[]
\CommentTok{\# Split the data into training (80\%) and test set (20\%)}
\FunctionTok{set.seed}\NormalTok{(}\DecValTok{123}\NormalTok{)}
\NormalTok{training.samples }\OtherTok{\textless{}{-}}\NormalTok{ data.c2}\SpecialCharTok{$}\NormalTok{cluster }\SpecialCharTok{\%\textgreater{}\%}
  \FunctionTok{createDataPartition}\NormalTok{(}\AttributeTok{p =} \FloatTok{0.8}\NormalTok{, }\AttributeTok{list =} \ConstantTok{FALSE}\NormalTok{)}
\NormalTok{train.data }\OtherTok{\textless{}{-}}\NormalTok{ data.c2[training.samples, ]}
\NormalTok{test.data }\OtherTok{\textless{}{-}}\NormalTok{ data.c2[}\SpecialCharTok{{-}}\NormalTok{training.samples, ]}
\end{Highlighting}
\end{Shaded}

\hypertarget{normalize-data-1}{%
\subsubsection{Normalize data}\label{normalize-data-1}}

\begin{Shaded}
\begin{Highlighting}[]
\CommentTok{\# Estimate preprocessing parameters}
\NormalTok{preproc.param }\OtherTok{\textless{}{-}}\NormalTok{ train.data }\SpecialCharTok{\%\textgreater{}\%} 
  \FunctionTok{preProcess}\NormalTok{(}\AttributeTok{method =} \FunctionTok{c}\NormalTok{(}\StringTok{"center"}\NormalTok{, }\StringTok{"scale"}\NormalTok{))}
\CommentTok{\# Transform the data using the estimated parameters}
\NormalTok{train.transformed }\OtherTok{\textless{}{-}}\NormalTok{ preproc.param }\SpecialCharTok{\%\textgreater{}\%} \FunctionTok{predict}\NormalTok{(train.data)}
\NormalTok{test.transformed }\OtherTok{\textless{}{-}}\NormalTok{ preproc.param }\SpecialCharTok{\%\textgreater{}\%} \FunctionTok{predict}\NormalTok{(test.data)}
\end{Highlighting}
\end{Shaded}

\hypertarget{compute-lda-1}{%
\subsubsection{Compute LDA}\label{compute-lda-1}}

\begin{Shaded}
\begin{Highlighting}[]
\NormalTok{model }\OtherTok{\textless{}{-}} \FunctionTok{lda}\NormalTok{(cluster}\SpecialCharTok{\textasciitilde{}}\NormalTok{., }\AttributeTok{data =}\NormalTok{ train.transformed)}
\NormalTok{model}
\end{Highlighting}
\end{Shaded}

\begin{verbatim}
## Call:
## lda(cluster ~ ., data = train.transformed)
## 
## Prior probabilities of groups:
## -0.158113883008418   6.16644143732834 
##              0.975              0.025 
## 
## Group means:
##                      Salegrow   saleproft     Newsale    createst    Mechtest
## -0.158113883008418 -0.0421968 -0.03730924 -0.06884791 -0.04332918 -0.04315877
## 6.16644143732834    1.6456753  1.45506023  2.68506851  1.68983814  1.68319184
##                       absttest    mathtest
## -0.158113883008418 -0.05332308 -0.04969867
## 6.16644143732834    2.07959996  1.93824821
## 
## Coefficients of linear discriminants:
##                  LD1
## Salegrow  -2.3012237
## saleproft -3.1539512
## Newsale   -1.0298809
## createst   1.5560344
## Mechtest   1.3197588
## absttest   0.9053936
## mathtest   5.0205975
\end{verbatim}

\hypertarget{make-predictions-1}{%
\subsection{2. Make predictions}\label{make-predictions-1}}

\begin{Shaded}
\begin{Highlighting}[]
\NormalTok{predictions }\OtherTok{\textless{}{-}}\NormalTok{ model }\SpecialCharTok{\%\textgreater{}\%} \FunctionTok{predict}\NormalTok{(test.data)}
\FunctionTok{names}\NormalTok{(predictions)}
\end{Highlighting}
\end{Shaded}

\begin{verbatim}
## [1] "class"     "posterior" "x"
\end{verbatim}

\hypertarget{model-accuracy-1}{%
\subsection{3. Model accuracy}\label{model-accuracy-1}}

\begin{Shaded}
\begin{Highlighting}[]
\FunctionTok{mean}\NormalTok{(predictions}\SpecialCharTok{$}\NormalTok{class}\SpecialCharTok{==}\NormalTok{test.transformed}\SpecialCharTok{$}\NormalTok{cluster)}
\end{Highlighting}
\end{Shaded}

\begin{verbatim}
## [1] 1
\end{verbatim}

Dilihat dari model akurasi dapat disimpulkan bahwa metode hierachical
clustering dengan jarak euclidean + metode single linkage lebih akurat
dibanding metode k-means.

\#simpan kluster

\begin{Shaded}
\begin{Highlighting}[]
\NormalTok{clusterSales }\OtherTok{\textless{}{-}} \FunctionTok{cbind}\NormalTok{(data, }\AttributeTok{Cluster =}\NormalTok{ km.res}\SpecialCharTok{$}\NormalTok{cluster)}
\FunctionTok{head}\NormalTok{(clusterSales)}
\end{Highlighting}
\end{Shaded}

\begin{verbatim}
##   Salegrow saleproft Newsale createst Mechtest absttest mathtest Cluster
## 1     93.0      96.0    97.8        9       12        9       20       2
## 2     88.8      91.8    96.8        7       10       10       15       2
## 3     95.0     100.3    99.0        8       12        9       26       2
## 4    101.3     103.8   106.8       13       14       12       29       1
## 5    102.0     107.8   103.0       10       15       12       32       1
## 6     95.8      97.5    99.3       10       14       11       21       2
\end{verbatim}

\begin{Shaded}
\begin{Highlighting}[]
\NormalTok{group }\OtherTok{\textless{}{-}} \FunctionTok{cutree}\NormalTok{(res.hc, }\AttributeTok{k=}\DecValTok{2}\NormalTok{)}
\NormalTok{HierarchySales }\OtherTok{\textless{}{-}} \FunctionTok{cbind}\NormalTok{(data, }\AttributeTok{Group =} \FunctionTok{as.factor}\NormalTok{(group))}
\FunctionTok{head}\NormalTok{(HierarchySales)}
\end{Highlighting}
\end{Shaded}

\begin{verbatim}
##   Salegrow saleproft Newsale createst Mechtest absttest mathtest Group
## 1     93.0      96.0    97.8        9       12        9       20     1
## 2     88.8      91.8    96.8        7       10       10       15     1
## 3     95.0     100.3    99.0        8       12        9       26     1
## 4    101.3     103.8   106.8       13       14       12       29     1
## 5    102.0     107.8   103.0       10       15       12       32     1
## 6     95.8      97.5    99.3       10       14       11       21     1
\end{verbatim}

\begin{Shaded}
\begin{Highlighting}[]
\NormalTok{HierarchySales }\SpecialCharTok{\%\textgreater{}\%}
  \FunctionTok{group\_by}\NormalTok{(Group) }\SpecialCharTok{\%\textgreater{}\%}
  \FunctionTok{summarise\_all}\NormalTok{(}\StringTok{"mean"}\NormalTok{)}
\end{Highlighting}
\end{Shaded}

\begin{verbatim}
## # A tibble: 2 x 8
##   Group Salegrow saleproft Newsale createst Mechtest absttest mathtest
## * <fct>    <dbl>     <dbl>   <dbl>    <dbl>    <dbl>    <dbl>    <dbl>
## 1 1         98.7      106.    103.     11.1     14.1     10.5     29.3
## 2 2        111.       122     115.     18       20       15       51
\end{verbatim}

\#\#iv. Buatkan analisis diskriminan untuk memperoleh: \#a. Prosentase
ketepatan klasifikasi 50 slaesman ke hasil iii.a. maupun ke hasil iii.b.

\#b. fungsi diskriminan berdasarkan hasil iii.a. maupun ke hasil iii.b.

\begin{Shaded}
\begin{Highlighting}[]
\FunctionTok{lda}\NormalTok{(Cluster}\SpecialCharTok{\textasciitilde{}}\NormalTok{., }\AttributeTok{data =}\NormalTok{ clusterSales)}
\end{Highlighting}
\end{Shaded}

\begin{verbatim}
## Call:
## lda(Cluster ~ ., data = clusterSales)
## 
## Prior probabilities of groups:
##    1    2 
## 0.58 0.42 
## 
## Group means:
##    Salegrow saleproft   Newsale createst Mechtest  absttest mathtest
## 1 104.07931 113.80345 105.90690 12.82759 15.68966 11.413793 37.13793
## 2  91.88095  96.70476  98.53333  9.00000 12.09524  9.380952 19.57143
## 
## Coefficients of linear discriminants:
##                  LD1
## Salegrow  -0.2355171
## saleproft -0.2280978
## Newsale   -0.2413533
## createst   0.1330236
## Mechtest   0.2773746
## absttest   0.1525134
## mathtest   0.1723275
\end{verbatim}

\begin{Shaded}
\begin{Highlighting}[]
\FunctionTok{lda}\NormalTok{(Group}\SpecialCharTok{\textasciitilde{}}\NormalTok{., }\AttributeTok{data =}\NormalTok{ HierarchySales)}
\end{Highlighting}
\end{Shaded}

\begin{verbatim}
## Call:
## lda(Group ~ ., data = HierarchySales)
## 
## Prior probabilities of groups:
##    1    2 
## 0.98 0.02 
## 
## Group means:
##    Salegrow saleproft  Newsale createst Mechtest absttest mathtest
## 1  98.71429  106.3082 102.5551 11.08163 14.06122 10.46939 29.32653
## 2 110.80000  122.0000 115.3000 18.00000 20.00000 15.00000 51.00000
## 
## Coefficients of linear discriminants:
##                  LD1
## Salegrow  -0.4387502
## saleproft -0.3523443
## Newsale   -0.1098142
## createst   0.3371347
## Mechtest   0.4602539
## absttest   0.3701557
## mathtest   0.5176323
\end{verbatim}

\end{document}
